\documentclass{article}
\usepackage{amsmath}
\usepackage[frenchb]{babel}
\usepackage[utf8]{inputenc}
\usepackage[T1]{fontenc}
\usepackage{hyperref}
\usepackage[usenames, dvipsnames]{color}

\title{Simulateur de netlist}
\author{Corentin Cadiou}
\newcommand{\code}[1]{\colorbox{White}{\texttt{#1}}}


\begin{document}
  \maketitle
\section{Information diverses}
\subsection{Simulateur}
\label{sec:simulateur}

Le simulateur de netlist a été écrit en OCaml. Ses sources sont entre
autre disponibles sur Github
(\url{https://github.com/cphyc/sysDig/}). Le rôle de chacun des
fichiers source se trouve dans le tableau \ref{tab:1}.
\begin{table}
  \noindent\begin{tabular}{|l|p{0.9\textwidth}|}
    \hline
    \textbf{Fichier source} & \textbf{Utilisation} \\
    \hline
    graph.ml & utlitaire pour gérer les graphes\\
    interaction.ml & module permettant l'interaction entre l'utilisateur et le programme \\
    myocamlbuild.ml & configuration de ocamlbuild \\
    netlist.ml & ouverture de la netlist avec gestion des erreurs \\
    netlist\_ast.ml & syntaxe abstraite de la netlist \\
    netlist\_lexer.mll & analyseur lexical de netlist \\
    netlist\_parser.mly & parseur de netlist \\
    netlist\_printer.ml & fonctions pour afficher le contenu d'une
    netlist \\
    ram.ml & module pour gérer la RAM \\
    rom.ml & module pour gérer la ROM \\
    rom\_comm.ml & interface de communication entre rom.ml et le reste du
    programme \\
    rom\_lexer.mll & analyseur lexical de fichier de ROM, se référer a la
    section \ref{sec:rom} \\
    scheduler.ml & ordonnanceur de netlist utilisant un tri topologique
    sur l'arbre de dépendance des équations \\
    simulator.ml & interprète la netlist \\
    main.ml & fichier principal qui ordonnance puis exécute la netlist\\
    \hline
  \end{tabular}
  \caption{\label{tab:1}Contenu des fichiers sources}
\end{table}

\subsection{Compilation}
Pour compiler, il suffit de se placer dans le dossier où se trouvent
les sources et de lancer la commande \code{make}.

\section{Structure du programme}
Le programme s'exécute de la manière suivante (simplifiée) :
\begin{enumerate}
  \item ouverture du fichier de netlist ;
  \item analyse lexical puis syntaxique du fichier ;
  \item ordonnancement des équations selon un \emph{tri topologique} ;
  \item affichage de l'ordonnancement du programme et fin du
    simulateur si l'option \texttt{-v} a été cochée, sinon continue ;
  \item mise en place de la ROM
    \begin{enumerate}
    \item ouverture du fichier de ROM ;
    \item analyse lexicale du fichier qui crée un tableau réprésentant
      la ROM (voir \ref{sec:rom}) ;
    \end{enumerate}
  \item initialisation de la RAM (voir \ref{sec:ram}) et de
    l'environnement (voir \ref{sec:environnement});
  \item exécution du programme \texttt{n} fois avec $\texttt{n} = 1$ par
    défaut et peut être modifié \emph{via} l'option \texttt{-n
      <valeur>}. Pour chaque itération, les opérations suivantes
    s'exécutent :
    \begin{enumerate}
      \item lecture des variables d'entrées depuis \texttt{stdin} et
        stockage dans la table de hashage \texttt{env} ;
      \item évaluation des équations \label{evaluation};
      \item affichage des variables de sortie ;
      \item mise à jour de l'environnement pour les variables de type
        registre et pour la RAM \label{eoc};
    \end{enumerate}
\end{enumerate}
\subsection{Environnement}
\label{sec:environnement}

L'environnement est représenté par une table de hashage dont les clés
sont les identifiants des variables et les valeurs sont de type
\texttt{value = VBitArray of bool array | VBit of bool}. Le choix de la
table de hashage a été effectué pour obtenir un temps (amorti)
d'écriture, de recherche et de suppression en $\mathcal{O}(1)$. Des
tests avec une map ont été effectués, celle-ci se trouve être plus
lente dans le cas général. 


\subsection{RAM}
\label{sec:ram}

\subsubsection{Représentation et initialisation}
La RAM est représentée en mémoire par un \code{bool array array ref}
Ocaml de longueur $2^\texttt{RAM add. size}$. La constante \texttt{RAM
add. size} est déterminée par la fonction \code{val create :
Netlist\_ast.program -> unit} qui effectue une pré-lecture de la
netlist. Cette lecture ignore toutes les équations sauf celles
contenant une instruction \code{RAM <address size> <word size>}. La
fonction initialise alors le tableau représentant la RAM de longueur
$2^\texttt{RAM add. size}$, chaque case faisant \texttt{RAM word
  size} de long.

\subsubsection{Accès à la RAM}
Les lectures de la RAM sont réalisées de manière immédiate (point \ref{evaluation})tandis que
les écritures sont faites en fin de cycle (point \ref{eoc}). Pour chaque instruction
d'écriture dans la RAM, il faut donc ``s'en rappeler'' et l'éxécuter
en fin de cycle. Pour des soucis de performance, j'ai choisi de
stocker les écritures demandées dans la RAM dans une pile de type
\code{(int*bool array) Stack.t} qui stocke des données sous forme de
tableau de booléens ainsi que l'adresse d'écriture. 

En fin de cycle, une itération est faite sur la pile puis la pile est vidée.

\subsection{Registre}
\label{sec:registre}

Les registres sont eux aussi ajoutés à une pile (pour ce cas précis,
une pile est $30\%$ plus rapide qu'une table de hashage) au fur et à
mesure qu'ils sont rencontrés dans la netlist. En fin de cycle, une
itération est effectuée sur la pile puis celle-ci est vidée.

\subsection{ROM}
\label{sec:rom}

La ROM est représentée en mémoire par un \code{bool array array ref}
Ocaml de longueur $2^\texttt{ROM add. size}$. Elle est
chargée automatiquement à partir du fichier \code{.rom} correspondant
au fichier \code{.net} chargé (\texttt{ma\_netlist.net} charge automatiquement
\texttt{ma\_netlist.rom} quand ce fichier existe). Le fichier
\code{.rom} doit suivre la syntaxe suivante :
\begin{verbatim}
ADDRESS SIZE   <adress size>
WORD SIZE      <word size>

<address 1> <value 1>
…
<address n> <value n>
\end{verbatim}
Les éléments entourés par des chevrons sont des valeurs en
binaire respectant l'expression régulière suivante \code{['0' '1']+}. Les espaces et les tabulations n'ont pas de valeur
syntaxique. Chaque paire addresse-valeur doit être séparée par au
moins un saut de ligne.

L'analyse du fichier \code{.rom} est effectuée par un lexer
(\texttt{rom\_lexer.ml}) qui initialise le tableau représentant la rom
dans la mémoire et le rempli au fur et à mesure de la lecture du
fichier décrivant la ROM.

L'accès à la ROM se fait en consultant la bonne case du tableau
directement, c'est-à-dire en temps constant.

\section{Difficultés rencontrées}

Lors de l'écriture du simulateur, la plus grosse difficulté a été de
créer des fonctions permettant de convertir les valeurs d'entrées
(\texttt{stdin} et \texttt{ROM}) en \texttt{bool array}.

C'est parce que la gestion de \texttt{string} en OCaml n'est pas
forcément évidente que le choix d'un analyseur lexical a été fait pour
lire la ROM. Cet analyseur a l'avantage d'avoir une syntaxe très
claire et d'être concis et facile à écrire.

\end{document}